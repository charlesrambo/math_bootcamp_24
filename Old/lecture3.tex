\documentclass{beamer}
\usepackage{pgffor,pgfmath}
\usepackage{lipsum}
\usepackage{multicol}
\usetheme{ucla}
\usepackage{graphicx}
\usepackage{listings}
\usepackage{verbatim}
\usepackage{tikz}
\linespread{1.5}
\usepackage{comment}

\usepackage{amsmath, amsthm, amssymb, latexsym}

%\newtheorem{definition}{Definition}

\title{Lecture 3}
\author{Charles Rambo}
\institute{UCLA Anderson School of Management}
\date{2023}
\location{Los Angeles, California}

% Turn on slide numbers:
\showSlideNumber{}

\AtBeginSection[]
{
    \begin{frame}
        \frametitle{Table of Contents}
        \tableofcontents[currentsection]
    \end{frame}
}


\begin{document}

\insertTitleSlide

\section{Sequences and Series}

\subsection{Sequences}

\begin{frame}
\frametitle{Sequences}
\begin{Definition}
A sequence $\displaystyle(a_n)_{n = 1}^\infty$ is said to {\bf converge}, if there is a value $a$ in $\mathbb{R}$ which has the property that: For all $\epsilon > 0$, there exists an integer $N$ such that $n\geq N$ implies that $|a_n - a| < \epsilon$.  We often write
$$
a_n\to a\qquad\text{or}\qquad\lim_{n\to\infty} a_n = a
$$
when $\displaystyle(a_n)_{n = 1}^\infty$ converges to $a$. If $\displaystyle(a_n)_{n = 1}^\infty$ does not converge, then it {\bf diverges}. 
\end{Definition}
\end{frame}

\frame{
\frametitle{Sequences Example}
\begin{Example}
Determine which of the following sequences converge/diverge. If the sequence converges, find its limit.
\begin{enumerate}
\item[(a)] $\displaystyle a_n = \frac{1}{n}$
\item[(b)] $\displaystyle b_n = \sqrt{n}$
\item[(c)]  $\displaystyle c_n = (-1)^n$
\item[(d)] $\displaystyle d_n = 1 + \frac{(-1)^n}{n}$
\end{enumerate} 
\end{Example}
}

\begin{frame}[fragile]
\frametitle{Python Sequences Example}
\begin{Example}
Use Python to graph the four sequences in the previous example. Graph them on separate subplots, and for the sequences that converge use horizontal lines to show their respective limits.
\end{Example}

\end{frame}

\begin{frame}[fragile]
\frametitle{Python Sequences Example Solution}
\begin{multicols}{2}
{\tiny
\linespread{0.8}
\begin{verbatim*}

# Import modules 
import numpy as np
import matplotlib.pyplot as plt

# Use latex
plt.rcParams['text.usetex'] = True
# Use Seaborn style
plt.style.use('seaborn')

# Define functions
a = lambda n: 1/n
b = lambda n: np.sqrt(n)
c = lambda n: (-1)**n
d = lambda n: 1 + (-1)**n/n

# Define the limits
a_lim, d_lim = 0, 1

# Get the n-values
n_vals = list(range(1, 21))

# Get the sequence values
a_vals = [a(n) for n in n_vals]
b_vals = [b(n) for n in n_vals]
c_vals = [c(n) for n in n_vals]
d_vals = [d(n) for n in n_vals]

# Set up subplots
fig, ax = plt.subplots(2, 2, sharey = True, 
   figsize = (10, 6))

# Plot a_n and its limit
ax[0, 0].scatter(n_vals, a_vals)
ax[0, 0].axhline(y = a_lim, color = 'r', 
   linestyle = 'dashed')
ax[0, 0].set_xlabel(r'$n$')
ax[0, 0].set_ylabel(r'$a_n$')

# Plot b_n and its limit
ax[0, 1].scatter(n_vals, b_vals, label = r'$b_n$')
ax[0, 1].set_xlabel(r'$n$')
ax[0, 1].set_ylabel(r'$b_n$')

# Plot c_n and its limit
ax[1, 0].scatter(n_vals, c_vals)
ax[1, 0].set_xlabel(r'$n$')
ax[1, 0].set_ylabel(r'$c_n$')

# Plot d_n and its limit
ax[1, 1].scatter(n_vals, d_vals)
ax[1, 1].axhline(y = d_lim, color = 'r', 
   linestyle = 'dashed')
ax[1, 1].set_xlabel(r'$n$')
ax[1, 1].set_ylabel(r'$c_n$')

plt.suptitle(r'Sequence Plots')

# Save the figure
plt.savefig(r'[location on machine]')
plt.show()

\end{verbatim*}
}
\end{multicols}
\end{frame}

\begin{frame}
\frametitle{Python Sequences Example Result}
\begin{center}
\includegraphics[scale = 0.4]{ex5.png}
\end{center}
\end{frame}

\begin{frame}
\frametitle{Triangle Inequality}
For any real numbers $x$, $y$, and $z$,
$$
|x - y| \leq |x -z| + |z - y|.
$$
\end{frame}

\frame{
\frametitle{Sequences}
\begin{Theorem}
\begin{itemize}
\item[(a)] The sequence $\displaystyle(a_n)_{n = 1}^\infty$ converges to $a$ in  $\mathbb{R}$ if and only if for every $\epsilon > 0$, we have $a_n$ in the interval $(a - \epsilon, a + \epsilon)$ for all but finitely many $n$.
\item[(b)] If $\displaystyle(a_n)_{n = 1}^\infty$ converges to both $a$ and $a$, then $a = b$.
\item[(c)] If $\displaystyle(a_n)_{n = 1}^\infty$ converges, then it is bounded. That is, convergence of $\displaystyle(a_n)_{n = 1}^\infty$ implies there exists a real number $N$ such that $|a_n| \leq N$ for all $n$.
\end{itemize}
\end{Theorem}
}

\frame{
\frametitle{Sequence Properties} 
\begin{Theorem} 
Suppose $\displaystyle(a_n)_{n = 1}^\infty$ and $\displaystyle(b_n)_{n = 1}^\infty$ are real numbered sequences and
$$
\lim_{n\to\infty} a_n = a\qquad\text{and}\qquad\lim_{n\to\infty} b_n = b.
$$
Let $\alpha$ and $\beta$ be real constants.
\begin{enumerate}
\item[(a)] $\displaystyle\lim_{n\to\infty}\left(\alpha a_n + \beta b_n\right) = \alpha a + \beta b$
\item[(b)] $\displaystyle\lim_{n\to\infty} a_n b_n = ab$
\item[(c)] $\displaystyle\lim_{n\to\infty} \frac{1}{a_n} = \frac{1}{a}$ if $a\neq 0$.
\end{enumerate}
\end{Theorem}
}

\begin{frame}
\frametitle{Monotonic Sequences}

\begin{Definition}
\begin{itemize}
\item A real sequence $(a_n)_{n = 1}^\infty$ is {\bf monotonically increasing} if $a_n \leq a_{n + 1}$ for all $n$.
\item A real sequence $(a_n)_{n = 1}^\infty$ is {\bf monotonically decreasing} if $a_n \geq a_{n + 1}$ for all $n$.
\end{itemize}
\end{Definition}

\begin{Theorem}
Suppose that  $(a_n)_{n = 1}^\infty$ is monotonic. Then it converges if and only if it is bounded.
\end{Theorem}
\end{frame}

\subsection{Series}

\begin{frame}
\frametitle{Series}
\begin{Definition}
Consider a series $\displaystyle S = \sum_{k = 1}^\infty a_k$. Its {\bf $\boldsymbol n$-th partial sum} is $\displaystyle S_n = \sum_{k = 1}^n a_k$. The series $S$ {\bf converges} if the sequence $(S_n)_{n = 1}^\infty$ converges, and it {\bf diverges} otherwise.
 \end{Definition}
 \end{frame}
 
 \begin{frame}[t]
 \frametitle{Example}
 \begin{Example}
 For what values of $r$ does the geometric series $\displaystyle\sum_{k = 1}^\infty r^{k - 1}$ converge?
 \end{Example}
 \end{frame}
 
 \begin{frame}
 \frametitle{Geometric Series}
 The geometric series is extremely important in finance. Remember these formulas. 
 $$
 \sum_{k= 1}^n a r^{k -1} = \frac{a(1 - r^n)}{1 - r}
 $$
 and
 $$
  \sum_{k = 1}^\infty a r^{k -1} =\begin{cases} \frac{a}{1 - r}, & |r| < 1\\ DNE,	& |r| \geq 1\end{cases}
 $$
  \end{frame}
  
  \begin{frame}[t]
   \frametitle{Convergent Series}
   \begin{Example}
   Show $\displaystyle\sum_{k = 1}^\infty \frac{1}{k(k + 1)}$ converges. 
   \end{Example}
  
  \end{frame}
 
 \begin{frame}
 \frametitle{Divergence Test}
 \begin{itemize}
\item If $\displaystyle\lim_{k\to\infty} a_k \neq 0$, then $\displaystyle\sum_{k = 1}^\infty a_k$ diverges.  
\item If $\displaystyle\lim_{k\to\infty} a_k =  0$, then $\displaystyle\sum_{k = 1}^\infty a_k$ may or may not converge.
\end{itemize}
 \end{frame}
 
 \begin{frame}
 \frametitle{Property of Series}
 
 \begin{Theorem} 
 Suppose that $\displaystyle\sum_{k = 1}^\infty a_k$ and $\displaystyle\sum_{k = 1}^\infty b_k$ converge. For any real constants $\alpha$ and $\beta$
 $$
 \sum_{k = 1}^\infty\left( \alpha a_k + \beta b_k\right) =   \alpha \sum_{k = 1}^\infty a_k + \beta \sum_{k = 1}^\infty b_k .
 $$
  \end{Theorem} 
 \end{frame}
 
 
 \begin{frame}
 \frametitle{Dominating Series}
 
 \begin{Theorem}
 If a series $\displaystyle\sum_{k = 1}^\infty b_k$ dominates a series $\displaystyle\sum_{k = 1}^\infty a_k$ in the sense that for all sufficiently large $k$, $|a_k| \leq b_k$, then converge of $\displaystyle\sum_{k = 1}^\infty b_k$ implies converge of $\displaystyle\sum_{k = 1}^\infty a_k$.
 \end{Theorem}
 
 \end{frame}
 
 \begin{frame}[t]
  \frametitle{Dominating Series Example}
  \begin{Example} 
  Show that the series $\displaystyle\sum_{k = 1}^\infty \frac{\sin k}{2^k}$ converges.
  \end{Example}
 
 \end{frame}
 
 \begin{frame}
  \frametitle{Integral Test}
  
  \begin{Theorem}[Integral Test]
\begin{enumerate}
\item[(a)] If $|a_k| \leq f(x)$ for all sufficiently large $k$ and all $x$ in the interval $(k - 1, k]$, then convergence of $\displaystyle\int_0^\infty f(x)\ dx$ implies convergence of $\displaystyle\sum_{k = 1}^\infty a_k$.
\item[(b)] If $|f(x)| \leq a_k$ for all sufficiently large $k$ and all $x$ in the interval $[k, k+1)$ then diverges of $\displaystyle\int_0^\infty f(x)\ dx$ implies divergence of $\displaystyle\sum_{k = 1}^\infty a_k$
\end{enumerate}
\end{Theorem}
  
  \end{frame}
  
  \begin{frame}[t]
  \frametitle{Example (p-Series)}
  \begin{Example}
  Prove that $\displaystyle\sum_{k = 1}^\infty \frac{1}{k^p}$ converges for $p > 1$ and diverges for $p\leq 1$.
  \end{Example}
  
  \end{frame}
  
  \begin{frame}
  \frametitle{Alternating Series Test}
  
  \begin{Theorem} 
Suppose the alternating series $\displaystyle\sum_{k = 1}^\infty (-1)^{k + 1} b_k$ is such that
  $$
  b_k \geq 0 \qquad\text{and}\qquad b_{k + 1} \leq b_k
  $$
 for sufficiently large $k$. Then the series converges if $\displaystyle\lim_{n\to\infty} b_k = 0$.
  \end{Theorem}
  \end{frame}
   
  \begin{frame}[t]
 \frametitle{Alternating Series Test Example}
 \begin{Example}
 Show that the harmonic series $\displaystyle\sum_{k = 1}^\infty \frac{(-1)^{k + 1}}{k}$ converges.
 \end{Example}
 
 \end{frame}
 
 
 \begin{frame}
  \frametitle{Absolutely and Conditionally Convergent Series}
  
  \begin{Definition}
  A series $\displaystyle\sum_{k = 1}^\infty a_k$ is {\bf absolutely convergent} if $\displaystyle\sum_{k = 1}^\infty |a_k|$ converges. A series is {\bf conditionally convergent} if $\displaystyle\sum_{k = 1}^\infty a_k$ converges but $\displaystyle\sum_{k = 1}^\infty |a_k|$ does not.
  \end{Definition}
  For example, the harmonic series
  $$
  \sum_{k = 1}^\infty \frac{(-1)^{k + 1}}{k}
  $$
  is conditionally convergent but not absolutely. 
  
 \end{frame}
 
 \begin{frame}
   \frametitle{Ratio Test}
   \begin{Theorem}
   \begin{enumerate}
   \item[(a)] If $\displaystyle\lim_{k\to\infty} \left|\frac{a_{k + 1}}{a_k}\right| = L < 1$, then the series $\displaystyle\sum_{k = 1}^\infty a_k$ is absolutely convergent.
   \item[(b)] If $\displaystyle\lim_{n\to\infty}\left|\frac{a_{k + 1}}{a_k}\right| = L > 1$, then the series $\displaystyle\sum_{k = 1}^\infty a_k$  is divergent.
   \item[(c)] If $\displaystyle\lim_{n\to\infty}\left|\frac{a_{k + 1}}{a_k}\right| = L = 1$, then the test fails.
   \end{enumerate}
   \end{Theorem}
 \end{frame}
 
 \begin{frame}[t]
  \frametitle{Ratio Test Example} 
  \begin{Example}
What can be said about the convergence of $\displaystyle\sum_{k = 1}^\infty (-1)^k\frac{k!}{k^k}$?
\end{Example}
\end{frame}

\subsection{Power Series}

\begin{frame}
\frametitle{Power Series}
\begin{Definition}
A {\bf power series} centered at $c$ is a series of the form
$$
\sum_{k = 0}^\infty a_k (x - c)^k.
$$
If the series converges for $|x - c| < R$ and diverges for $|x - c| > R$, then $R$ is the {\bf radius of convergence}. The {\bf interval of convergence} $I$ is the set of all $x$ values where the series converges. 
\end{Definition}
{\bf Remark:} We assume $0^0 = 1$ within our power series, so the power series always converges at $x = c$.
\end{frame}

\begin{frame}[t]
\frametitle{Power Series Example}
\begin{Example}
Find the radius and interval of convergence of the series $\displaystyle\sum_{k = 0}^\infty \frac{(-3)^k (x + 1)^k}{\sqrt{k + 1}}$.
\end{Example}

\end{frame}

\begin{frame}
\frametitle{Differentiation and Integration of Power Series}
\begin{Theorem}
If the power series $f(x) = \displaystyle\sum_{k = 0}^\infty a_k (x - c)^k$ has radius of convergence $R > 0$, then both
\begin{enumerate}
\item[(a)] $\displaystyle f\ ' (x) = \sum_{k = 1}^\infty k a_k (x - c)^{k - 1}$ and
\item[(b)] $\displaystyle\int f(x)\ dx = C + \sum_{k = 0}^\infty a_k \frac{(x - c)^{k + 1}}{k + 1}$
\end{enumerate}
have radii of convergence $R$.
\end{Theorem}
\end{frame}

\begin{frame}[t]
\frametitle{Differentiation of Power Series Example}
\begin{Example}
$\displaystyle\sum_{k = 1}^\infty \frac{k}{1.10^k}$ = 
\end{Example}
\end{frame}

\begin{frame}
\frametitle{Taylor's Theorem}
\begin{Theorem}
Suppose
$$
f(x) = \sum_{k = 0}^\infty a_k (x - c)^k\qquad\text{for}\qquad |x - c| < R.
$$
Then
$$
a_k = \frac{f\ ^{(k)}(c)}{k!}.
$$
\end{Theorem}
\end{frame}

\begin{frame}
\frametitle{Popular Taylor Series Centered at Zero}
%\begin{multicols}{2}
{\small
\begin{itemize}
\item $\displaystyle \frac{1}{1 - x} = \sum_{k = 0}^\infty x^k$ for $x\in (-1, 1)$
\item $\displaystyle e^x = \sum_{k = 0}^\infty \frac{x^k}{k!}$ for $x\in\mathbb{R}$
\item $\displaystyle\sin x = \sum_{k = 0}^\infty (-1)^k \frac{x^{2k +1}}{(2k + 1)!}$ for $x\in\mathbb{R}$
\item $\displaystyle\cos x = \sum_{k = 0}^\infty (-1)^k \frac{x^{2k}}{(2k)!}$ for $x\in\mathbb{R}$
\item $\displaystyle\arctan x = \sum_{k = 0}^\infty (-1)^k \frac{x^{2k + 1}}{2k+ 1}$ for $x\in [-1, 1]$
\end{itemize}
}
%\end{multicols}

\end{frame}

\begin{frame}[t]
\frametitle{Taylor's Theorem Example}
\begin{Example}
Prove $\displaystyle\arctan x = \sum_{k = 0}^\infty (-1)^k \frac{x^{2k + 1}}{2k+ 1}$ for $x\in [-1, 1]$.
\end{Example}
\end{frame}

%\begin{frame}[t]
%\frametitle{Taylor's Theorem Example}
%\begin{Example}
%Find a power series, centered at 0 for $\displaystyle\int_0^x \sin(t^2)\ dt$.
%\end{Example}
%\end{frame}

\section{Time Value of Money}

\begin{frame}
\frametitle{Time Value of Money}

For a time $t$ cash flow $C_t$ discounted at rate $r$, the {\it present value} is
$$
PV = \frac{C_t}{(1 + r)^t}.
$$
The time $T$ {\it future value} is
$$
FV_T = PV (1 + r)^T = C_t (1 + r)^{T- t}.
$$
\end{frame}

\begin{frame}
\frametitle{Compound Interest}
\small 
We assumed that interest is compounded once per unit of time. However, if it is compounded $n$ times the formulas become
$$
PV = \frac{C_t}{\left(1 + \frac{r}{n}\right)^{nt}}\qquad\text{and}\qquad FV_T = PV \left(1 + \frac{r}{n}\right)^{nT}.
$$
Since
$$
\lim_{n\to\infty} \left(1 + \frac{r}{n}\right)^{nt} = e^{rt}
$$
for continuous compounding (i.e. $n = \infty$) the formulas are
$$
PV = C_t e^{-rt} \qquad\text{and}\qquad PV_T = PV  e^{rT} = PV  e^{r(T - t)}
$$
\end{frame}

\begin{frame}
\frametitle{Multiple Cash Flows} 
Typically we have a sequence of cash flows $C_0, C_1, C_2,\ldots, C_T$ the {\it net present value} of these cash flows discounted at the constant rate $r$ is
$$
NPV = C_0 + \frac{C_1}{1 + r} + \frac{C_2}{(1 + r)^2} + \ldots + \frac{C_n}{(1 + r)^T}.
$$
\end{frame}

\begin{frame}[fragile]
\frametitle{Time Value of Money Python Example}
\small
\begin{Example}
Suppose cash flows are as shown in the table below.
\begin{center}
\begin{tabular}{| l | c c c c c |}
\hline
Time			&	0		&	1	&	2	&	3	&	4\\\hline
Cash Flow	&	-100		&	50	&	20	&	70	&	10\\\hline
\end{tabular}
\end{center}
Calculate the net present value given a continuously compounded discount rate of 5\%.
\end{Example}
\begin{multicols}{2}
{
\linespread{0.8}
{\bf\small Solution.}  
{\tiny
\begin{verbatim*}
# Import module
import numpy as np

# Record rate
rate = 0.05

# Record time of cash flows
time = np.array([0, 1, 2, 3, 4])

# Record cash flows
cash_flows = np.array([-100, 50, 20, 70, 10])

# Get the NPV
NPV = np.sum(cash_flows * np.exp(-rate * time))

print(f'The NPV of the cash flows is {NPV:.2f}.')
\end{verbatim*}
}}

$NPV \approx 34.10$
\end{multicols}
\end{frame}


\begin{frame}[t]
\frametitle{Time Value of Money Example}
\small
\begin{Example}
Jain borrows \$1,000,000 to purchase a house. The loan is for thirty years and her first payment is one month from when she initially borrows the money. If her annualized rate is 12\%, what will be her monthly payments? Ignore fees.
\end{Example}

\end{frame}

\begin{frame}
\frametitle{Growing Payments}
Suppose 1 is payed at time $1$, and payments increase at a rate of $g$ each subsequent period until a final payment of $(1 + g)^{n -1}$ is made at time $n$. If cash flows are discounted at rate $r$, then the NPV of the cash flows is
$$
\frac{1 - \left(\frac{1 + g}{1 + r}\right)^n}{r - g}.
$$
\end{frame}

\begin{frame}[t]
\frametitle{Growing Payments Example}
\tiny
\begin{Example}
Calculate the NPV of the series of end-of-year cash flows. Assume 
\begin{itemize}
\item \$100 is payed in the first year,
\item each subsequent year payments increase by 5\%, 
\item the final payment is made at the end of year ten, and
\item the discount rate is 8\%.
\end{itemize} 
\end{Example}

\end{frame}

\end{document}
